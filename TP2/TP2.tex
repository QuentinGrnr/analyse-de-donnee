%! Author = quent
%! Date = 09/02/2025

\documentclass{article}
\usepackage{graphicx}
\usepackage{amsmath}
\usepackage{verbatim}  % Pour les blocs de code
\usepackage{float}     % Pour le positionnement des images
\usepackage{subcaption} % Pour afficher plusieurs images en sous-figures

\title{\textbf{Analyse des Donn\'ees - TP2}}
\author{Quentin Garnier}
\date{2025}

\begin{document}

    \maketitle

    \section{Criminalité aux USA}
    \subsection{Exploration élémentaire}
    \subsubsection{Identification des valeurs atypiques}

    Pour détecter des valeurs atypiques dans notre jeu de données, nous analysons d'abord les statistiques descriptives. En observant la variable \textbf{robbery}, nous constatons que :

    \begin{itemize}
        \item Le \textbf{3ème quartile (Q3)} est de 155.85.
        \item La valeur \textbf{maximale} est de 472.0.
    \end{itemize}

    Afin de valider cette observation, nous afficons le \textbf{boxplot} de la variable \textit{robbery}. Comme prévu, nous voyons un point isolé au-dessus de la moustache supérieure, ce qui confirme la présence d'une valeur extrême.

    Cette valeur pourrait correspondre à un état où les taux de vols avec violence sont exceptionnellement élevés. Une anlyse plus approfondie permettrait d'identifier les causes sous-jacentes de cette anomalie.

    \begin{figure}[H]
        \centering
        \includegraphics[width=0.5\linewidth]{img/robbery_data_boxplot}
        \caption{BoxPlot des données de robbery}
    \end{figure}

    L’analyse de la matrice de corrélation nous permet d’identifier des relations entre les différentes variables criminelles.

    \subsubsection{Corrélations fortes}
    Certaines variables montrent une corrélation élevée, indiquant une relation sgnificative :
    \begin{itemize}
        \item \textbf{Rape et Assault} : 0.74 (forte corrélaton positive)
        \item \textbf{Burglary et Larceny} : 0.79 (forte corrélation positive)
        \item \textbf{Rape et Burglary} : 0.71 (forte corrélation positivbe)
    \end{itemize}

    \subsubsection{Corrélations faibles ou inexistantes}
    Certaines variables ne montrent pas de lien significatif :
    \begin{itemize}
        \item \textbf{Murder et uto} : 0.06
    \end{itemize}

    \subsubsection{Interprétation}
    L'analyse révèle deux grands groupes :
    \begin{itemize}
        \item \textbf{Les crimes violents} (Murder, Rape, Assault, Robbery) sont fortement corrélés entre eux.
        \item \textbf{Les crimes contre les biens} (Burglary, Larceny, Auto) montrent également une forte corrélation.
        \item Il y a peu de lien entre ces deux goupes
    \end{itemize}

    \subsection{Analyse en Composantes Principales (ACP)}

    \subsubsection{Réalisation de l’ACP}
    Nous effectuons l'ACP sur la table \texttt{crime}

    \subsubsection{Choix de la dimension}

    L’analyse des corrélations montre deux groupes principaux :
    \begin{itemize}
        \item \textbf{Crimes violents} : Murder, Rape, Assault, Robbery
        \item \textbf{Crimes contre les biens} : Burglary, Larceny, Auto
    \end{itemize}

    \subsubsection{Représentation des individus et vriables}

    \subsubsection{Variance des composantes}
    Le graphique ci-dessous montre la variance expliquée par chaque composante :
    \begin{figure}[H]
        \centering
        \includegraphics[width=0.5\linewidth]{img/plotcrimacp}
        \caption{BoxPlot des données de robbery}
    \end{figure}

    \textbf{Observation :} Les deux premières composantes expliquent l’essentielde la variance, les suivantes apportent peu d’information.

    \subsubsection{Confirmation du choix de la dimension}
    Les biplots permettent d’évaluer la répartition des individus et des variables :
    \begin{figure}[H]
        \centering
        \includegraphics[width=0.5\linewidth]{img/bowplotcrimeacpscore}
        \caption{BoxPlot des données de robbery}
    \end{figure}

    \begin{figure}[H]
        \centering
        \includegraphics[width=0.5\linewidth]{img/bigplotcrimeacp}
        \caption{BoxPlot des données de robbery}
    \end{figure}

    \begin{figure}[H]
        \centering
        \includegraphics[width=0.5\linewidth]{img/bigplotcrimeacp2}
        \caption{BoxPlot des données de robbery}
    \end{figure}
    \textbf{Conclusion :} Les composantes 1et 2 suffisent pour bien représenter les données. La troisième n'apporte pas d’information significative supplémentaire.
    \subsubsection{Identification des valeurs atypiques sur la troisième composante}

    D’après l’ACP, deux états apparaissent comme \textbf{atypiques} sur la troisième composante : \textbf{New York} et \textbf{Massachusetts}.

    Ces valeurs influencent potentiellement la définition des axes. Il est recommandé de vérifier si leur exclusion modifie l’interprétation des résultats.

    \subsubsection{Exclusion des valeurs atypiques}

    Après avoir exclu \textbf{New York} et \textbf{Massachusetts}, vici les nouveaux biplots :

    \begin{figure}[H]
        \centering
        \includegraphics[width=0.5\linewidth]{img/bigplotcrimeacp3}
        \caption{Biplot des composantes 1 et 2 après exclusion}
    \end{figure}

    \begin{figure}[H]
        \centering
        \includegraphics[width=0.6\linewidth]{img/bigplotcrimeacp4}
        \caption{Biplot des composantes 2 et 3 après exclusion}
    \end{figure}

    \textbf{Observations :}
    \begin{itemize}
        \item Le plan (\textbf{Comp.1, Comp.2}) reste similaire après l’exclusion.
        \item L’axe \textbf{Comp.3} est légèrement modifié,ce qui montre que ces états avaient une influence sur cette dimension.
    \end{itemize}

    \textbf{Conclusion :} L’exclusion n’affecte pas l’analyse globale mais réduit l’effet des valeurs extrêmes.

    \subsubsection{Interprétation des axes}
    L'axe 1 permet de distinguer les crimes violents des crimes contre les biens. Les états avec un score élevé sur cet axe ont un taux de crimes violents plus élevé, tandis que ceux avec un score faible ont un taux de crimes contre les biens plus élevé.
    L'axe 2 permet de distinguer les états avec un taux élevé de crimes violents et de crimes contre les biens. Les états avec un score élevé sur cet axe ont un taux de crimes violents plus élevé, tandis que ceux avec un score faible ont un taux de crimes contre les biens plus élevé.

    \section{Hôtels Méditerranéens}

    \subsection{ACP avec une variable qualitative supplémentaire}

    \subsubsection{Chargement des données}

    \subsubsection{Analyse des variables}

    Le jeu de données \texttt{hotels} contient 39 hôtels et 8 variables :

    \begin{itemize}
        \item \textbf{PAYS} : Variable qualitative indiquant le pays dimplantation (variable supplémentaire dans l'ACP).
        \item \textbf{ETOILE} : Nombre d’étoiles de l’hôtel (quantitative discrète).
        \item \textbf{CONFORT} : Niveau de confort noté sur une échelle (quantitative discrète).
        \item \textbf{CHAMBRE} : Nombre total de chambres (quantitative continue).
        \item \textbf{CUISINE} : Niveau d’équipement de la cuisine (quantitative discrète).
        \item \textbf{SPORT} : Niveau d’équipement sortif (quantitative discrète).
        \item \textbf{PLAGE} : Indicateur d’accès à une plage (quantitative discrète).
        \item \textbf{PRIX} : Prix moyen des chambres (quantitative continue).
    \end{itemize}

    \textbf{Individus analysés :} Chaque ligne du tableau correspond à un hôtel.

    \textbf{Statistiques descriptives :}

    \begin{figure}[H]
        \centering
        \includegraphics[width=0.6\linewidth]{img/summaryhotel}
        \caption{Biplot des composantes 2 et 3 après exclusion}
    \end{figure}

    \textbf{Covariances et corrélations :}

    \begin{figure}[H]
        \centering
        \includegraphics[width=0.6\linewidth]{img/covhotel}
        \caption{Biplot des composantes 2 et 3 après exclusion}
    \end{figure}

    On observe :
    \begin{itemize}
        \item Une \textbf{forte corrélation linéaire} entre \textbf{ÉTOILE et CONFORT} ($r = 0.63$).
        \item Une \textbf{corrélation notable} entre \textbf{ÉTOILE et CUISINE} ($r = 0.60$).
        \item Une \textbf{corrlation moyenne} entre \textbf{ÉTOILE et PRI} ($r = 0.54$), ce qui est attendu puisque les hôtels plus étoilés sont souvent plus chers.
        \item Peu de corrélations fortes entre les autres variables.
    \end{itemize}

    \textbf{Identification des liaisons non linéaires :}
    Si certaines relations ne sont pas linéaires, elles peuvent être visualisées avec :

    \begin{verbatim}
    pairs(hotels[,sapply(hotels, is.numeric)])
    \end{verbatim}

    \begin{figure}[H]
        \centering
        \includegraphics[width=0.6\linewidth]{img/pairhotel}
        \caption{Biplot des composantes 2 et 3 après exclusion}
    \end{figure}


    On observe des structures non linéaires, notamment entre :
    \begin{itemize}
        \item \textbf{PRIX et CUISINE}
        \item \textbf{SPORT et d'autres vaiables}
        \item \textbf{PLAGE et PRIX}
    \end{itemize}

    \subsubsection{Analyse en Composantes Principales}

    Les diagrammes des valeurs propres montrent l’inertie expliquée par chaque composante :

    \begin{figure}[H]
        \centering
        \includegraphics[width=0.6\linewidth]{img/inertie_expliquee}
        \caption{Inertie expliquée par composante}
    \end{figure}

    \begin{figure}[H]
        \centering
        \includegraphics[width=0.6\linewidth]{img/inertie_cumulee}
        \caption{Inertie cumulée expliquée}
    \end{figure}

    L’inertie cumulée dépasse **40\% après la troisième cmposante** mais reste loin du seuil de **80\%**.
    Le choix optimal dépend du compromis entre simplification et information retenue, mais **3 composantes principales semblent suffisantes pour un bon résumé des données**.

    \subsubsection{Représentation des individus et des variables}

    Les individus (hôtels) sont représentés sur les premiers axes factoriels :

    \begin{figure}[H]
        \centering
        \includegraphics[width=0.6\linewidth]{img/PCA1.}
        \caption{Projection des individus sur Dim 1 et Dim 2}
    \end{figure}

    \begin{figure}[H]
        \centering
        \includegraphics[width=0.6\linewidth]{img/PCA2}
        \caption{Projection des individus sur Dim 3 et Dim 4}
    \end{figure}

    Les variables sont représentées dans les cercles de corrélation :

    \begin{figure}[H]
        \centering
        \includegraphics[width=0.6\linewidth]{img/PCA4}
        \caption{Cercle des corrélations - Dim 1 et Dim 2}
    \end{figure}

    \begin{figure}[H]
        \centering
        \includegraphics[width=0.6\linewidth]{img/PCA3}
        \caption{Cercle des corrélations - Dim 3 et Dim 4}
    \end{figure}

    L’analyse des représentations montre une bonne séparation des hôtels par pays sur les premiers axes et une forte structuration des variables dans les cercles de corrélation.
    l'axe 1 représenterait plus les hôtels de luxe, tandis que l'axe 2 représenterait les hôtels orientés vers le confort et les équipements sportifs.
    A souligné que la Cuisine est totalement independante de l'axe 2 ce qui est assez logique, car la cuisine est un critère de choix pour les hôtels, mais n'est pas directement lié au confort ou aux équipements sportifs.

    \subsubsection{Utilisation de dimdesc}

    \begin{itemize}
        \item \textbf{Dimension 1} :
        \begin{itemize}
            \item Les variables continues les plus corréles sont \textbf{CUISINE} (0.89), \textbf{PRIX} (0.76) et \textbf{ETOILE} (0.71).
            \item La variable catégorielle \textbf{PAYS} est significative (\textit{p-value} = 0.006).
            \item Les pays \textbf{Portugal} et \textbf{Grèce} inluencent cette dimension.
        \end{itemize}

        \item \textbf{Dimension 2} :
        \begin{itemize}
            \item Les variables continues les plus corrélées sont \textbf{PLAGE} (0.67) et \textbf{SPORT} (0.67).
            \item La variable \textbf{PAYS} est aussi significative (\textit{p-value} = 0.009).
            \item Les pays \textbf{Tunisie} et \textbf{Maroc} influencent cette dimension.
        \end{itemize}

        \item \textbf{Dimension 3} :
        \begin{itemize}
            \item Les variables coninues les plus corrélées sont \textbf{CHAMBRE} (0.69) et \textbf{PRIX} (-0.48).
            \item La variable \textbf{PAYSl} est également significative (\textit{p-value} = 0.011).
            \item Le pays \textbf{Portugal} influence cette dimension.
        \end{itemize}
    \end{itemize}

    \subsubsection{Utilisation de coord.ellipse}
    \begin{figure}[H]
        \centering
        \includegraphics[width=0.6\linewidth]{img/PlotPCA1}
        \caption{Cercle des corrélations - Dim 3 et Dim 4}
    \end{figure}
    \begin{figure}[H]
        \centering
        \includegraphics[width=0.6\linewidth]{img/PlotPCA2}
        \caption{Cercle des corrélations - Dim 3 et Dim 4}
    \end{figure}
    On observe que :
    \begin{itemize}
        \item Les hôtels situés au \textbf{Portugal} et au \textbf{Maroc} sont plutot bien distincts des autres groupes.
        \item La \textbf{Tnisie} et la \textbf{Turquie} présentent une forte superposition, suggérant que leurs hôtels partagent des caractéristiques similaires.
    \end{itemize}
    \subsection{ACP avec variables qualitative et quantitative supplémentaires}
    Ayant un projet a rendre pour lundi 10 février pour mon universitée eramsus en Italie, je n'ai pu commencer a travailler sur ce projet qu'aujourd'hui, je n'ai donc pas eu le temps de faire cette partie du TP.

\end{document}
